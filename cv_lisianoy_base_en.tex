%% start of file `template.tex'.
%% Copyright 2006-2013 Xavier Danaux (xdanaux@gmail.com).
%
% This work may be distributed and/or modified under the
% conditions of the LaTeX Project Public License version 1.3c,
% available at http://www.latex-project.org/lppl/.


% possible options include:
% font size
% ('10pt', '11pt', '12pt')
% paper size
% ('a4paper', 'letterpaper', 'a5paper',
% 'legalpaper', 'executivepaper', 'landscape')
% font family
% ('sans' and 'roman')
\documentclass[11pt,a4paper,oneside,roman]{moderncv}

% moderncv themes
% style options
% 'casual' (default), 'classic', 'oldstyle' and 'banking'
\moderncvstyle{classic}
% color options:
% 'blue' (default), 'orange', 'green', 'red', 'purple', 'grey' and 'black'
\moderncvcolor{green}
% to set the default font, use '\sfdefault' for the default sans serif font,
% '\rmdefault' for the default roman one, or any tex font name
%\renewcommand{\familydefault}{\sfdefault}
% uncomment to suppress automatic page numbering for CVs longer than one page
%\nopagenumbers{}

% character encoding
\usepackage{fontspec}
\usepackage{libertine}
\usepackage{inconsolata}

% adjust the page margins
\usepackage[scale=0.75]{geometry}
% if you want to change the width of the column with the dates
%\setlength{\hintscolumnwidth}{3cm}

\AtBeginDocument{\hypersetup{colorlinks=true,urlcolor=blue}}

% For the 'classic' style
% If you want to force the width allocated to your name and avoid line breaks.
% Be careful though, the length is normally calculated to avoid any overlap
% with your personal info; use this at your own typographical risks...
%\setlength{\makecvtitlenamewidth}{10cm}

% personal data
\name{Aleksandr}{Lisianoi}
%\title{Resumé title}
% the "postcode city" and "country" arguments can be omitted or provided empty
%\address{street and number}{postcode city}{country}
% possible types: mobile, fixed or fax
%\phone[mobile]{0660~155~67~60}
%\phone[fixed]{+2~(345)~678~901}
%\phone[fax]{+3~(456)~789~012}
\email{all3fox@gmail.com}
%\homepage{www.johndoe.com}
\social[linkedin]{all3fox}
%\social[twitter]{}
\social[github]{all3fox}
\extrainfo{08.07.1993 (23 y/o)}
% '64pt' is the height the picture must be resized to,
% 0.4pt is the thickness of the frame around it (0pt for no frame),
% 'picture' is the name of the picture file
%\quote{Résumé composed \today}

%% to enalbe colored links
%% \AtBeginDocument{
%%     \hypersetup{colorlinks,urlcolor=cyan}
%% }

%\photo[100pt][0.4pt]{picture}
\begin{document}
\makecvtitle

\section{Education}
%\cventry{year--year}{Degree}{Institution}{City}
%        {\textit{Grade}}{Description}
\cventry{2015--2017}
        {Faculty of Informatics}
        {Vienna University of Technology}
        {}{}
        {Bachelor of Software and Information Engineering, ongoing.}
\cventry{2011--2015}
        {Faculty of Computational Mathematics}
        {Lomonosov Moscow State University}
        {}{}
        {Bachelor of Applied Mathematics and Computer Science, graduated.}
\section{Experience}
\cventry{2016}{Google Summer of Code}
        {Organization: jQuery, Inc}{}
        {}{Worked on the \href{https://github.com/jzaefferer/commitplease/}{commitplease} subproject. Added support for AngularJS commit message style, as well as \href{https://github.com/brigade/overcommit}{overcommit} and \href{https://github.com/typicode/husky}{husky} projects. The npm module is \href{https://www.npmjs.com/package/commitplease}{downloaded 11k/month}. (node.js/git internals)}
\cventry{2014--2015}{R\&D intern}
        {NDM Systems, Inc}
        {}{}
        {Developed a scraping system that mimics user activity and gathers video flows from the web. Extracted features from video flows, studied their impact on video recognition and used the results to help consumer network devices better prioritize internet traffic. \\ (Python/scikit-learn/Selenium/PostgreSQL)}

\section{Selected Coursework}
\cventry{2016}{GINP: Graphic Image Node Processor}
        {}{}{}{A desktop image processing application that treats images and image operations as nodes of a directed acyclic graph. Development in a small team, my main input are graph traversal algorithms that check graph structure, process images and cache image operation results. \\ (Java 8/JavaFX/Spring/JUnit/Maven/git)}

\cventry{2014}{3D Model of the Solar System}
        {}{}{}{Experienced a variety of 3D image rendering techniques. Enjoyed coding the basic algorithms: Kepler's and Newton's laws of motion, Gourard and Phong shading models, etc. \\ (C++11/OpenGL/Boost.Test/CMake/git)}
\section{Languages \small{(based on ILR scale)}}
\cvitemwithcomment{Russian}{Native or bilingual proficiency}{}
\cvitemwithcomment{English}{Full professional proficiency}
  {Exams:
    \href{http://www.cambridgeenglish.org/exams/advanced/}{CAE}~(grade~A),
    \href{http://www.ets.org/toefl}{TOEFL~iBT}~(115/120 and 118/120)
  }
\cvitemwithcomment{German}{Limited working proficiency}{}

\section{Exam Scores}
\cventry{26--11--2014}{GRE}
        {verbal 158/170, quantative 170/170, writing 3/6}
        {}{}{}
\cventry{3--11--2012}{SAT Mathematics II}{800/800}{}{}{}
\cventry{6--10--2012}{SAT Physics}{800/800}{}{}{}

%% \section{Vocational}
%% \cventry{2014--2015}
%%         {\href{https://www.coursera.org/course/algs4partI}{Algorithms Part I} and \href{https://www.coursera.org/course/algs4partII}{Algorithms Part II}}
%%         {Coursera}
%%         {(java)}{}{}
%% \cventry{2012}{Sprache -- Kultur -- Literatur}
%%         {\href{http://treffpunktsprachen.uni-graz.at/en/}
%%           {Summer course at University of Graz}}
%%         {Austria}{}{German language, Austrian culture and literature.}
\end{document}
%% end of file `template.tex'.
