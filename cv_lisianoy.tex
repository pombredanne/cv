%% start of file `template.tex'.
%% Copyright 2006-2013 Xavier Danaux (xdanaux@gmail.com).
%
% This work may be distributed and/or modified under the
% conditions of the LaTeX Project Public License version 1.3c,
% available at http://www.latex-project.org/lppl/.


% possible options include font size ('10pt', '11pt' and '12pt'), paper size ('a4paper', 'letterpaper', 'a5paper', 'legalpaper', 'executivepaper' and 'landscape') and font family ('sans' and 'roman')>
\documentclass[11pt,a4paper,sans]{moderncv}

% moderncv themes
\moderncvstyle{classic}                             % style options are 'casual' (default), 'classic', 'oldstyle' and 'banking'
\moderncvcolor{blue}                               % color options 'blue' (default), 'orange', 'green', 'red', 'purple', 'grey' and 'black'
%\renewcommand{\familydefault}{\sfdefault}         % to set the default font; use '\sfdefault' for the default sans serif font, '\rmdefault' for the default roman one, or any tex font name
%\nopagenumbers{}                                  % uncomment to suppress automatic page numbering for CVs longer than one page

% character encoding
\usepackage[utf8]{inputenc}
\usepackage[T2A]{fontenc}
\usepackage[english,russian]{babel}

% adjust the page margins
\usepackage[scale=0.75]{geometry}
%\setlength{\hintscolumnwidth}{3cm}                % if you want to change the width of the column with the dates
%\setlength{\makecvtitlenamewidth}{10cm}           % for the 'classic' style, if you want to force the width allocated to your name and avoid line breaks. be careful though, the length is normally calculated to avoid any overlap with your personal info; use this at your own typographical risks...

% personal data
\name{Александр}{Лисяной}
%\title{Resumé title}
% the "postcode city" and "country" arguments can be omitted or provided empty
%\address{street and number}{postcode city}{country}
\phone[mobile]{+7~(916)~822~6977} % possible types: mobile, fixed or fax
%\phone[fixed]{+2~(345)~678~901}
%\phone[fax]{+3~(456)~789~012}
\email{all3fox@gmail.com}
%\homepage{www.johndoe.com}
%\social[linkedin]{john.doe}
%\social[twitter]{jdoe}
\social[github]{all3fox}
\extrainfo{08.07.1993 (21 год), Москва}
% '64pt' is the height the picture must be resized to,
% 0.4pt is the thickness of the frame around it (0pt for no frame),
% 'picture' is the name of the picture file
%\photo[64pt][0.4pt]{picture}
%\quote{Some quote}

%% to enalbe colored links
%% \AtBeginDocument{
%%     \hypersetup{colorlinks,urlcolor=cyan}
%% }

\begin{document}
\makecvtitle

\section{Образование}
\cventry{2011--2015}{Прикладная математика и информатика}
        {\href{http://cs.msu.ru/}{ВМК МГУ}}{Москва}
        {Бакалавр (4~курс)}
        {Занимаюсь машинным обучением
          (\href{http://en.wikipedia.org/wiki/Pattern_recognition}{распознавание объектов},
          \href{http://en.wikipedia.org/wiki/Statistical_classification}{классификация})
        }
\cventry{2008--2010}{Математика и экономика}
        {\href{http://liceum1535.ru/}{Лицей 1535}}
        {Москва}{}{}
%\cventry{year--year}{Degree}{Institution}{City}{\textit{Grade}}{Description}

%% \section{Master thesis}
%% \cvitem{title}{\emph{Title}}
%% \cvitem{supervisors}{Supervisors}
%% \cvitem{description}{Short thesis abstract}

%% \section{Experience}
%% \subsection{Vocational}
%% \cventry{year--year}{Job title}{Employer}{City}{}{General description no longer than 1--2 lines.\newline{}%
%% Detailed achievements:%
%% \begin{itemize}%
%% \item Achievement 1;
%% \item Achievement 2, with sub-achievements:
%%   \begin{itemize}%
%%   \item Sub-achievement (a);
%%   \item Sub-achievement (b), with sub-sub-achievements (don't do this!);
%%     \begin{itemize}
%%     \item Sub-sub-achievement i;
%%     \item Sub-sub-achievement ii;
%%     \item Sub-sub-achievement iii;
%%     \end{itemize}
%%   \item Sub-achievement (c);
%%   \end{itemize}
%% \item Achievement 3.
%% \end{itemize}}
%% \cventry{year--year}{Job title}{Employer}{City}{}{Description line 1\newline{}Description line 2}
%% \subsection{Miscellaneous}
%% \cventry{year--year}{Job title}{Employer}{City}{}{Description}

\section{Знание языков}
\cvitemwithcomment{Английский}{свободное владение}
                  {Сданы экзамены:
                    \href{https://ru.wikipedia.org/wiki/Certificate_in_Advanced_English}{CAE}~(grade~A),
                    \href{https://ru.wikipedia.org/wiki/TOEFL}{TOEFL~iBT}~(118~из~120)
                  }
\cvitem{Немецкий}{базовое владение}
\cvitem{Русский}{родной язык}

\section{Компьютерные навыки}
\cvitem{Языки}{Python, С/C++, MATLAB, Java}
\cvitem{Бибилиотеки}{
  \href{http://scikit-learn.org/stable/}{scikit--learn},
  \href{http://www.csie.ntu.edu.tw/~cjlin/libsvm/}{libsvm},
  \href{http://www.csie.ntu.edu.tw/~cjlin/liblinear/}{liblinear}
}
\cvitem{Тестирование}{unittest (Python), unittest (MATLAB), JUnit (Java)}
\cvitem{Прочее}{
  \href{http://git-scm.com/}{git},
  \href{http://www.cmake.org/}{cmake},
  \href{http://www.gnu.org/software/emacs/}{Emacs},
  \href{https://www.archlinux.org/}{Arch Linux}
}

\section{Кратко подходящие навыки}
\cvlistitem{Опыт администрирования роутера на
  \href{http://wiki.mikrotik.com/wiki/MikroTik_RouterOS}{MikroTik Router OS}
}
\cvlistitem{Опыт работы с \href{https://www.wireshark.org/}{Wireshark}}
\cvlistitem{Базовые представления об
  \href{http://en.wikipedia.org/wiki/OSI_model}{OSI} модели и компьютерных сетях
}

%% \section{Interests}
%% \cvitem{hobby 1}{Description}
%% \cvitem{hobby 2}{Description}
%% \cvitem{hobby 3}{Description}

%% \section{Extra 1}
%% \cvlistitem{Item 1}
%% \cvlistitem{Item 2}
%% \cvlistitem{Item 3. This item is particularly long and therefore normally spans over several lines. Did you notice the indentation when the line wraps?}

%% \section{Extra 2}
%% \cvlistdoubleitem{Item 1}{Item 4}
%% \cvlistdoubleitem{Item 2}{Item 5\cite{book1}}
%% \cvlistdoubleitem{Item 3}{Item 6. Like item 3 in the single column list before, this item is particularly long to wrap over several lines.}

%% \section{References}
%% \begin{cvcolumns}
%%   \cvcolumn{Category 1}{\begin{itemize}\item Person 1\item Person 2\item Person 3\end{itemize}}
%%   \cvcolumn{Category 2}{Amongst others:\begin{itemize}\item Person 1, and\item Person 2\end{itemize}(more upon request)}
%%   \cvcolumn[0.5]{All the rest \& some more}{\textit{That} person, and \textbf{those} also (all available upon request).}
%% \end{cvcolumns}

\end{document}


%% end of file `template.tex'.
